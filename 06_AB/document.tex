% !TeX encoding = UTF-8
% !TeX spellcheck = de_DE

\documentclass[DIV=calc,numbers=noenddot]{scrartcl}

\usepackage[ngerman]{babel}
\usepackage{fontspec}
\usepackage{amsmath}
\usepackage{graphicx}
\usepackage{comment}
\usepackage{hyperref}
\usepackage{calc}
\usepackage[colorinlistoftodos]{todonotes}

\newcounter{blatt}
\setcounter{blatt}{6} % Nummer des Aufgabenblattes
\renewcommand{\thesection}{Exercise \arabic{blatt}.\arabic{section}}
\renewcommand{\thesubsection}{Aufgabenteil \arabic{subsection}}
\renewcommand{\thesubsubsection}{(\alph{subsubsection})}

\title{GWV - Grundlagen der Wissensverarbeitung Blatt \arabic{blatt}}
\author{Julian Tobergte, Melanie Budde,\\Maximilian Bauregger, Mohammad Oslani}
\date{\today}

\begin{document}
	\maketitle
	\section{Constraints}
		\subsection{}
			Das constraint network ist angehängt.
		\subsection{}
			Unser ,,informaler`` Ansatz wäre es, sich zuerst, zB. für Spalte D1,  ein zufälliges Wort auszusuchen und nun die Spalten A1 - A3 mit Worten aus der Liste füllen. Ergeben sich dadurch in D2 und D3 keine erlaubten Wörter, so müssen andere Worte für die Zeilen A1 - A3 gewählt werden. Sind hierfür wiederum keine möglich, so muss ein anderes Wort für D1 gewählt werden. Dies kann man nun so lange durchprobieren, bis eine passende Belegung gefunden ist.
\end{document}