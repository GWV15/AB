% !TeX encoding = UTF-8
% !TeX spellcheck = de_DE

\documentclass[DIV=9,numbers=noenddot]{scrartcl}

\usepackage[ngerman]{babel}
\usepackage{fontspec}
\usepackage{amsmath}
\usepackage{graphicx}
\usepackage{comment}
\usepackage{hyperref}
\usepackage{calc}
\usepackage[colorinlistoftodos]{todonotes}

\newcounter{blatt}
\setcounter{blatt}{8} % Nummer des Aufgabenblattes
\renewcommand{\thesection}{Exercise \arabic{blatt}.\arabic{section}}
\renewcommand{\thesubsection}{\arabic{subsection}.}
\renewcommand{\thesubsubsection}{(\alph{subsubsection})}

\title{GWV - Grundlagen der Wissensverarbeitung}
\subtitle{Tutorial \arabic{blatt}}
\author{Julian Tobergte, Melanie Budde,\\Maximilian Bauregger, Mohammad Oslani}
\date{\today}
\setcounter{section}{1}

\begin{document}
	\maketitle
	\section{Language Modelling}
		\setcounter{subsection}{1}
		\subsection{}
			Nehmen wir z.B. den folgenden generierten Satz: "`Linux gründende neue Herausforderung nicht mehr als Original-Sun-Speicherriegel. in diesen Ankündigungen gibt es auch mit der darauf, der."' Grammatikalisch ist hier einiges im Argen. Es existieren sinnvolle Wortkombinationen, wie "`nicht mehr als"' oder "`in diesen Ankündigungen"'. Eine Satzstruktur oder gar Sinn ist nicht vorhanden. Dies ergibt sich, da nur direkt benachbarte Worte miteinander verknüpft sind, Grammatik sich aber über den gesamten Satz erstreckt.
	\section{Diagnosis (cont.)}
	\section{Bayesian Probabilities}
\end{document}
